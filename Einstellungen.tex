% Vorlage für WABs der Provadis Hochschule
% basiert auf "Universität Ulm Praktikumsbericht Vorlage"
% von Max Sch.
% CC BY 4.0
%
% Einstellungen.tex
%
% Hier werden allgemeine Einstellungen festgelegt.

\documentclass[a4paper,ngerman,12pt,bibliography=totoc,listof=totoc,numbers=noendperiod,openany,twoside,headinclude=false,titlepage=firstiscover,abstract=true,headsepline=true,footsepline=true,headings=big]{scrreprt}
%Startet das Dokument und regelt gleichzeitig Formatierungen wie Schriftgröße, Papierformat, Zahlenart, Sprache usw.
% 12pt - die Standardschriftgröße
% openany - Kapitel dürfen auch auf der "Rückseite" starten, Leerseiten werden vermieden
% parskip=full - Leerzeilen zwischen Absätzen statt Einrückung
% twoside - Rückseiten werden mit bedruckt, Seitenzahlen und Binderände werden entsprechen angepasst; Alternative: oneside

%Packages sind Erweiterungen des eigentlichen Programms und erleichtern Abläufe, Darstellungen, Formatierungen usw.
%Im Folgenden sind die Packages installiert, welche ich zum erstellen einer Bachelorarbeit einmal benötigt wurden.
%Es kann beliebig erweitert/reduziert werden.

%%%%%%%%%%%%%%%%%%%%%%%%%%%%%%%%%%%%%%%%%%%%%%%%%%%%%%%%%
%%%%%%%%%%%%%%%%%%%%%%%%%%%%%%%%%%%%%%%%%%%%%%%%%%%%%%%%%

% Paket für Times Schrift
%\usepackage{mathptmx}

\usepackage{charter}

\usepackage[ngerman]{babel}
% Paket für Deutsche Sprache (Übersetzungen von Chapter zu Kapitel, 
% richtige Umlaute, richtige % Silbentrennung)
% siehe auch http://de.wikipedia.org/wiki/Babel-System

\usepackage{titling}
%Hilft im Dokument auf den deklarierten Autor und Titel zuzugreifen 

\usepackage[style=numeric-comp,backend=biber,doi=false,isbn=false,maxnames=3,sorting=none]{biblatex}
\usepackage{csquotes}
\addbibresource{Referenzen.bib}
%Regelt die Erstellung des Literaturverzeichnisses sowie die Zitation im gesamten Bericht
%Greift gleichzeitig auf die Referenzen.bib genannte BibTex datei zu, in welcher die Quellen aufgeführt sind.

\usepackage{url}
%erzeugt schönere URLs (vorallem in den Quellen wichtig)

\usepackage{float}
%Hilft beim Anordnen von Bildern, Grafiken und Tabellen

%\usepackage{abstract}
%ermöglicht das einfügen eines Abstract (Nicht relevant fürs Praktikum)

\usepackage{booktabs}
%ermöglicht das erstellen von schöneren Tabellen 

\usepackage{subfigure}
%lässt den Benutzer mehrere Grafiken/Bilder in einer Abbildung darstellen

\usepackage{setspace}
% Paket um den Zeilenabstand zu ändern
\onehalfspacing
% Zeilenabstand auf 1,5-fach setzen

\usepackage{pdfpages}
%ermöglicht das einbinden von eigenständigen PDF Dokumenten 

\usepackage{siunitx} 
\sisetup{locale = DE}
%vereinfacht den benötigten Syntax zum Benutzen von SI Einheiten und Mathematischen Gleichungen mit dem deutschen Standart

\usepackage{tikz}
%Freies Anordnen von Bildern und Text in Relation zueinander, sowie Möglichkeiten Bilder zu bearbeiten
%Verfügt auserdem über die Möglichkeit Grafiken und Diagramme in Latex zu erstellen

\usepackage[format=plain,labelfont=bf]{caption}
\captionsetup[table]{position=above}
%verschönert die Abbildungs- und Tabellenbeschriftungen. Für Tabellen werden die Beschriftungen über der Tabelle angezeigt

\usepackage{icomma}
%Für den korrekten Abstand der Kommas in Dezimalzahlen und in Gleichungen

\usepackage{amsmath}
%elementares Paket für Gleichungen. Verschönert diese und erleichtert die Benutzung von Gleichungen.

\usepackage{graphicx}
%mehr Optionen beim Einbinden und anordnen von Bildern und Grafiken

\usepackage{setspace}
%Für den richtigen Zeilenabstand

%\usepackage[left=3.5cm, right=2.3cm, top=3cm, bottom=2.5cm]{geometry} %links mehr
\usepackage[left=3cm,right=2cm,top=3cm, bottom=3cm]{geometry}
%Regelt die Seitenabstände

\usepackage{enumitem}

\usepackage[acronym, toc]{glossaries}

\renewcommand*{\glspostdescription}{}
%Entfernt die Nummerierung im Abkürzungsverzeichnis

\usepackage{listings}
\definecolor{codegreen}{rgb}{0,0.6,0}
\definecolor{codegray}{rgb}{0.5,0.5,0.5}
\definecolor{codepurple}{rgb}{0.58,0,0.82}
\definecolor{backcolour}{rgb}{0.95,0.95,0.92}
\lstdefinestyle{myStyle}{
    backgroundcolor=\color{backcolour},   
    commentstyle=\color{codegreen},
    keywordstyle=\color{magenta},
    numberstyle=\tiny\color{codegray},
    stringstyle=\color{codepurple},
    basicstyle=\ttfamily\footnotesize,
    breakatwhitespace=false,         
    breaklines=true,                 
    keepspaces=true,                 
    numbers=left,       
    numbersep=5pt,                  
    showspaces=false,                
    showstringspaces=false,
    showtabs=false,                  
    tabsize=2,
}

\linespread{1.5}
\flushbottom

\iffalse % don’t use titlesec.sty with KOMA script classes
\usepackage{titlesec}
%% \titleformat{⟨command⟩}[⟨shape⟩]{⟨format⟩}{⟨label⟩}{⟨sep⟩}{⟨before⟩}[⟨after⟩]
\titleformat{name=\chapter}[display]
{\usekomafont{chapter}}
{\raggedleft\chaptertitlename\ {\textcolor{gray}{\fontsize{60}    {65}\selectfont\thechapter}}}
{20pt}{}[]
\fi

%%%%%%%%%%%%%%%%%%%%%%%%%%%%%%%%%%%%% ToC DEPTH LEVEL
\setcounter{secnumdepth}{3} % number subsubsection
\setcounter{tocdepth}{3} % list subsubsection


\addtokomafont{chapterprefix}{\raggedleft}

%%%%%%%%%%%%%%%%%%%%%%%%%%%%%%%%%%%%% PREFACE AND BODY STYLING
\def\preface{
    \pagenumbering{roman}
    \doublespacing
}

\def\body{
    %\cleardoublepage
    \linespread{1.5}
    \pagenumbering{arabic}
    \pagestyle{headings}
}

\def\abstract{
    \begin{center}{
        \large\bfseries Kurzzusammenfassung}
    \end{center}
    \normalsize
    \normalfont
    \linespread{1.5}
    }{\cleardoublepage}
\def\endabstract{
  \par
}

\newenvironment{acknowledgements}{
   \cleardoublepage
    \begin{center}{
        \large \bf Danksagung}
    \end{center}
    \normalsize
    \linespread{1.5}
    }{\cleardoublepage}
\def\endacknowledgements{
  \par
}

%%%%%%%%%%%%%%%%%%%%%%%%%%%%%%%%%%%%%%%%%%%%%%%%%%%%%%%%%
%%%%%%%%%%%%%%%%%%%%%%%%%%%%%%%%%%%%%%%%%%%%%%%%%%%%%%%%%
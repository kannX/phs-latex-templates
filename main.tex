% Vorlage für WABs der Provadis Hochschule
% basiert auf "Universität Ulm Praktikumsbericht Vorlage"
% von Max Sch.
% CC BY 4.0
%
% main.tex
%
% Das ist das zentrale Dokument.


%persönliche Infos
% Vorlage für Praktikumsberichte
%
% Berichtsinfos.tex
%
% Hier werden individuelle Inforamtionen festgelegt.

%persönliche Daten
\newcommand{\fullname}{Max Mustermann}
\newcommand{\email}{max.mustermann@stud-provadis-hochschule.de}
\newcommand{\matnr}{A123}
\newcommand{\telMobil}{+49 1234 567890}
\newcommand{\fakultaet}{Informatik und Wirtschaftsinformatik}

%Daten zum Unternehmen
\newcommand{\titel}{Textvorlage für wissenschaftliche Arbeiten\\
	Titel und Untertitel der Arbeit}
\newcommand{\unternehmen}{Musterfirma}

\newcommand{\abgabedatum}{xx.xx.xxxx}

% Daten zu Betreuern
\newcommand{\refFirst}{Prof. Dr. First Last}
\newcommand{\refSecond}{Dr. First Last}

%Deklariert den Namen des Authors, sowie den Namen der generierten pdf Datei
\title{Wissenschaftlich angeleitet Berufspraxis - \unternehmen}
\author{\fullname}
\date{\today}

\documentclass[a4paper,ngerman,12pt,bibliography=totoc,listof=totoc,numbers=noendperiod,openany,twoside,headinclude=false,titlepage=firstiscover,abstract=true,headsepline=true,footsepline=true,headings=big]{scrreprt}
%Startet das Dokument und regelt gleichzeitig Formatierungen wie Schriftgröße, Papierformat, Zahlenart, Sprache usw.
% 12pt - die Standardschriftgröße
% openany - Kapitel dürfen auch auf der "Rückseite" starten, Leerseiten werden vermieden
% parskip=full - Leerzeilen zwischen Absätzen statt Einrückung
% twoside - Rückseiten werden mit bedruckt, Seitenzahlen und Binderände werden entsprechen angepasst; Alternative: oneside

% \RedeclareSectionCommand[
%   beforeskip=-32pt
% ]{chapter}
% Optionaler Befehl zum editieren von dem margin top von Kapiteln

%allgemeine Einstellungen
% Vorlage für WABs der Provadis Hochschule
% basiert auf "Universität Ulm Praktikumsbericht Vorlage"
% von Max Sch.
% CC BY 4.0
%
% Einstellungen.tex
%
% Hier werden allgemeine Einstellungen festgelegt.

\documentclass[a4paper,ngerman,12pt,bibliography=totoc,listof=totoc,numbers=noendperiod,openany,twoside,headinclude=false,titlepage=firstiscover,abstract=true,headsepline=true,footsepline=true,headings=big]{scrreprt}
%Startet das Dokument und regelt gleichzeitig Formatierungen wie Schriftgröße, Papierformat, Zahlenart, Sprache usw.
% 12pt - die Standardschriftgröße
% openany - Kapitel dürfen auch auf der "Rückseite" starten, Leerseiten werden vermieden
% parskip=full - Leerzeilen zwischen Absätzen statt Einrückung
% twoside - Rückseiten werden mit bedruckt, Seitenzahlen und Binderände werden entsprechen angepasst; Alternative: oneside

%Packages sind Erweiterungen des eigentlichen Programms und erleichtern Abläufe, Darstellungen, Formatierungen usw.
%Im Folgenden sind die Packages installiert, welche ich zum erstellen einer Bachelorarbeit einmal benötigt wurden.
%Es kann beliebig erweitert/reduziert werden.

%%%%%%%%%%%%%%%%%%%%%%%%%%%%%%%%%%%%%%%%%%%%%%%%%%%%%%%%%
%%%%%%%%%%%%%%%%%%%%%%%%%%%%%%%%%%%%%%%%%%%%%%%%%%%%%%%%%

% Paket für Times Schrift
%\usepackage{mathptmx}

\usepackage{charter}

\usepackage[ngerman]{babel}
% Paket für Deutsche Sprache (Übersetzungen von Chapter zu Kapitel, 
% richtige Umlaute, richtige % Silbentrennung)
% siehe auch http://de.wikipedia.org/wiki/Babel-System

\usepackage{titling}
%Hilft im Dokument auf den deklarierten Autor und Titel zuzugreifen 

\usepackage[style=numeric-comp,backend=biber,doi=false,isbn=false,maxnames=3,sorting=none]{biblatex}
\usepackage{csquotes}
\addbibresource{Referenzen.bib}
%Regelt die Erstellung des Literaturverzeichnisses sowie die Zitation im gesamten Bericht
%Greift gleichzeitig auf die Referenzen.bib genannte BibTex datei zu, in welcher die Quellen aufgeführt sind.

\usepackage{url}
%erzeugt schönere URLs (vorallem in den Quellen wichtig)

\usepackage{float}
%Hilft beim Anordnen von Bildern, Grafiken und Tabellen

%\usepackage{abstract}
%ermöglicht das einfügen eines Abstract (Nicht relevant fürs Praktikum)

\usepackage{booktabs}
%ermöglicht das erstellen von schöneren Tabellen 

\usepackage{subfigure}
%lässt den Benutzer mehrere Grafiken/Bilder in einer Abbildung darstellen

\usepackage{setspace}
% Paket um den Zeilenabstand zu ändern
\onehalfspacing
% Zeilenabstand auf 1,5-fach setzen

\usepackage{pdfpages}
%ermöglicht das einbinden von eigenständigen PDF Dokumenten 

\usepackage{siunitx} 
\sisetup{locale = DE}
%vereinfacht den benötigten Syntax zum Benutzen von SI Einheiten und Mathematischen Gleichungen mit dem deutschen Standart

\usepackage{tikz}
%Freies Anordnen von Bildern und Text in Relation zueinander, sowie Möglichkeiten Bilder zu bearbeiten
%Verfügt auserdem über die Möglichkeit Grafiken und Diagramme in Latex zu erstellen

\usepackage[format=plain,labelfont=bf]{caption}
\captionsetup[table]{position=above}
%verschönert die Abbildungs- und Tabellenbeschriftungen. Für Tabellen werden die Beschriftungen über der Tabelle angezeigt

\usepackage{icomma}
%Für den korrekten Abstand der Kommas in Dezimalzahlen und in Gleichungen

\usepackage{amsmath}
%elementares Paket für Gleichungen. Verschönert diese und erleichtert die Benutzung von Gleichungen.

\usepackage{graphicx}
%mehr Optionen beim Einbinden und anordnen von Bildern und Grafiken

\usepackage{setspace}
%Für den richtigen Zeilenabstand

%\usepackage[left=3.5cm, right=2.3cm, top=3cm, bottom=2.5cm]{geometry} %links mehr
\usepackage[left=3cm,right=2cm,top=3cm, bottom=3cm]{geometry}
%Regelt die Seitenabstände

\usepackage{enumitem}

\usepackage[acronym, toc]{glossaries}

\renewcommand*{\glspostdescription}{}
%Entfernt die Nummerierung im Abkürzungsverzeichnis

\usepackage{listings}
\definecolor{codegreen}{rgb}{0,0.6,0}
\definecolor{codegray}{rgb}{0.5,0.5,0.5}
\definecolor{codepurple}{rgb}{0.58,0,0.82}
\definecolor{backcolour}{rgb}{0.95,0.95,0.92}
\lstdefinestyle{myStyle}{
    backgroundcolor=\color{backcolour},   
    commentstyle=\color{codegreen},
    keywordstyle=\color{magenta},
    numberstyle=\tiny\color{codegray},
    stringstyle=\color{codepurple},
    basicstyle=\ttfamily\footnotesize,
    breakatwhitespace=false,         
    breaklines=true,                 
    keepspaces=true,                 
    numbers=left,       
    numbersep=5pt,                  
    showspaces=false,                
    showstringspaces=false,
    showtabs=false,                  
    tabsize=2,
}

\linespread{1.5}
\flushbottom

\iffalse % don’t use titlesec.sty with KOMA script classes
\usepackage{titlesec}
%% \titleformat{⟨command⟩}[⟨shape⟩]{⟨format⟩}{⟨label⟩}{⟨sep⟩}{⟨before⟩}[⟨after⟩]
\titleformat{name=\chapter}[display]
{\usekomafont{chapter}}
{\raggedleft\chaptertitlename\ {\textcolor{gray}{\fontsize{60}    {65}\selectfont\thechapter}}}
{20pt}{}[]
\fi

%%%%%%%%%%%%%%%%%%%%%%%%%%%%%%%%%%%%% ToC DEPTH LEVEL
\setcounter{secnumdepth}{3} % number subsubsection
\setcounter{tocdepth}{3} % list subsubsection


\addtokomafont{chapterprefix}{\raggedleft}

%%%%%%%%%%%%%%%%%%%%%%%%%%%%%%%%%%%%% PREFACE AND BODY STYLING
\def\preface{
    \pagenumbering{roman}
    \doublespacing
}

\def\body{
    %\cleardoublepage
    \linespread{1.5}
    \pagenumbering{arabic}
    \pagestyle{headings}
}

\def\abstract{
    \begin{center}{
        \large\bfseries Kurzzusammenfassung}
    \end{center}
    \normalsize
    \normalfont
    \linespread{1.5}
    }{\cleardoublepage}
\def\endabstract{
  \par
}

\newenvironment{acknowledgements}{
   \cleardoublepage
    \begin{center}{
        \large \bf Danksagung}
    \end{center}
    \normalsize
    \linespread{1.5}
    }{\cleardoublepage}
\def\endacknowledgements{
  \par
}

%%%%%%%%%%%%%%%%%%%%%%%%%%%%%%%%%%%%%%%%%%%%%%%%%%%%%%%%%
%%%%%%%%%%%%%%%%%%%%%%%%%%%%%%%%%%%%%%%%%%%%%%%%%%%%%%%%%

\input{Glossar}

\newacronym{Usw}{Usw.}{Und so weiter}

\begin{document}

% Vorlage für WABs der Provadis Hochschule
% basiert auf "Universität Ulm Praktikumsbericht Vorlage"
% von Max Sch.
% CC BY 4.0
%
% Titelseite.tex
%
% Hier wird die Titelseite erzeugt.


\begin{titlepage}

    % Logo Provadis Hochschule und ggf. Logo Arbeitgeber
    \includegraphics[height=2.06cm]{Bilder/Deckblatt/provadis-hochschule.pdf}
    \hfill
    % optionales Unternehmenslogo
    %\includegraphics[height=2.06cm]{Bilder/Deckblatt/Infraserv_Höchst_logo.pdf}

    \vspace*{1cm}

    \begin{singlespace}
        \begin{center}

            \normalsize
            Wiss. Kurzbericht / WAB / Bachelorarbeit / Masterarbeit

            \vspace*{2cm}

            \large

            \textbf{\titel}

            \vspace*{3cm}

            \normalsize
            Zur Veranstaltung / zur Erlangung des akademischen Grades\\
            ’Bachelor of Science’ B.Sc. / ’Master of Science’ M.Sc.\\
            im Studiengang ’XXX’

            \vspace*{2cm}

            vorgelegt dem Fachbereich \fakultaet \ der\\
            Provadis School of International Management and Technology\\
            von

            \vspace*{1cm}

            \fullname \\
            \matnr \\
            Studiengang mit Studiengruppe \\
            \email \\
            \telMobil

        \end{center}
    \end{singlespace}

    \normalsize
    \vfill % variabler vertikaler Abstand
    \begin{tabular}{@{}ll}
        Erstgutachter: & \refFirst \\[0.25cm]
        Zweitgutachter: & \refSecond \\[0.25cm]
        Ende der Bearbeitungsfrist: & \abgabedatum \\
    \end{tabular}

\end{titlepage}

% Ende der Datei


\preface
%Ab hier Seitenzahlen mit Römischen Ziffern

% auskommentieren, um den Sperrvermerk wegzulassen
% Vorlage für WABs der Provadis Hochschule
% basiert auf "Universität Ulm Praktikumsbericht Vorlage"
% von Max Sch.
% CC BY 4.0
%
% Sperrvermerk.tex
%
\addchap{Sperrvermerk}

Ein Sperrvermerk ist nur zulässig, wenn zwingend \emph{schützensnotwendige}, vertrauliche Informationen in der Arbeit enthalten sind. Dies ist normalerweise nur dann anzunehmen, wenn betriebsinterne Informationen des Arbeitsgebers \emph{ohne Generalisierung, Abstrahierung oder Pseudonymisierung wiedergegeben werden}.

Die Überprüfung der Quellen durch die Gutachter muss trotz Sperrvermerk möglich sein. Die mit der Begutachtung und Prüfung von der Hochschule beauftragten Personen (direkt und indirekt) müssen zur Nutzung der Inhalte der Prüfungsleistung berechtigt sein.

Eine Nutzung der (digitalen) Dokumente für Zwecke der Plagiatsüberprüfung muss erlaubt sein.

Sperrvermerke, die hierzu in Widerspruch stehen, sind wahlweise nichtig oder führen dazu, dass die Arbeit als nicht abgegeben angesehen wird / die Prüfungs\-leistung als nicht erbracht bewertet wird.

In begründeten Fällen (siehe oben) kann ein Sperrvermerk wie folgt aussehen:

\begin{verse}
Diese Arbeit wurde unter Verwendung von vertraulichen Informationen der Firma „XXX – bitte ersetzen“ angefertigt. Diese Arbeit darf Dritten ohne ausdrückliche Zustimmung der Firma „XXX – bitte ersetzen“ nicht zugänglich gemacht werden.

Alle Nutzungen für die Begutachtung durch die Hochschule und von ihr beauftragte Dienstleister sind explizit erlaubt. Zu der Begutachtung gehört auch die Prüfung auf Plagiate und die Verwendung künstlicher Intelligenz unter Nutzung entsprechender IT-Plattformen.
\end{verse}


% auskommentieren, um den Abstract/Kurzzusammenfassung wegzulassen
% Vorlage für WABs der Provadis Hochschule
% basiert auf "Universität Ulm Praktikumsbericht Vorlage"
% von Max Sch.
% CC BY 4.0
%
% Abstract.tex

\addcontentsline{toc}{chapter}{Abstract}
\begin{abstract}
Eine optionale Kurzzusammenfassung.

Bei englischsprachigen Arbeiten sollte auch eine Kurzzusammenfassung in deutscher Sprache angefügt werden.
\end{abstract}


\tableofcontents
%erstellt das vollautomatische Inhaltsverzeichnis

\listoffigures
%erstellt das Abbildungsverzeichnis

\listoftables
%erstellt das Tabellenverzeichnis

\glsaddall[types=main]
%Befehl, um alle Glossar Einträge ohne sie im body aufzurufen, auszugeben
\printglossary[title=Glossar, type=\glsdefaulttype]
%erstellt das Glossar

\printglossary[type=\acronymtype, title=Abkürzungsverzeichnis, nonumberlist]
%erstellt das Abkürzungsverzeichnis

\newpage

\body
%ab hier Seitenzahlen in Arabischen Zahlen

% Vorlage für WABs der Provadis Hochschule
% basiert auf "Universität Ulm Praktikumsbericht Vorlage"
% von Max Sch.
% CC BY 4.0
%
% 01_Einleitung.tex

\chapter{Einführung in Thema und Forschungsfrage}

\dictum[Sun Tzu]{If you know the enemy and you know yourself %\\
you need not fear the results of a hundred battles.}

Hier wird das Thema beschrieben. Es wird eine Fokussierung vorgenommen und eine überprüfbare Forschungsfrage herausgearbeitet.

Wie jeder Fließtext soll dieses Kapitel im Blocksatz geschrieben werden, um die Lesbarkeit zu erleichtern.

Selbstverständlich habe ich die Rechtschreibprüfung und die Grammatikprüfung angeschaltet und mache immer wieder eine Überprüfung, ob alles richtig ist.


\section{Abschnitt}

Beispiel für einen Abschnitt. \newline
Beispiel für eine Abkürzung: \gls{Usw}

\subsection{Unterabschnitt}

Beispiel für einen Unterabschnitt.

\subsubsection{Unter-Unterabschnitt}
Beispiel für einen Unter-Unterabschnitt.
% Vorlage für WABs der Provadis Hochschule
% basiert auf "Universität Ulm Praktikumsbericht Vorlage"
% von Max Sch.
% CC BY 4.0
%
% 02_Stand.tex

\chapter{Stand der Forschung}

In diesem Kapitel wird besonders intensiv mit der Literatur gearbeitet. Entsprechend werden hier sehr viele Zitate vorkommen. Ich werde die Quellenhinweise sofort einbauen, damit ich sie nicht vergesse. 

Dabei beachte ich die Vorgaben der Provadis School of International Management and Technology, die in den „Anforderungen an wissenschaftliche Hausarbeiten“ beschrieben sind.

Die Literatur wird in einem Verzeichnis am Ende der Arbeit aufgeführt.
Ebenso werden Quellen (Interviews, Internetseiten) in einem gesonderten Quellenverzeichnis genannt.

Literaturquellen werden in der Datei Referenzen.bib im BibTeX-Format eingetragen und im Text~\cite{lustigeCitation} zitiert. Gut gepflegte BibTeX-Einträge finden sich etwa in der DBLP\footnote{\url{https://dblp.uni-trier.de}} und in den Online-Bibliotheken großer Wissenschaftsverlage, wie der ACM Digital Library\footnote{\url{https://dl.acm.org}}, IEEE Xplore\footnote{\url{https://ieeexplore.ieee.org}}, Springer Link\footnote{\url{https://link.springer.com}}, Elsevier ScienceDirect\footnote{\url{https://www.elsevier.com/de-de/products/sciencedirect}}, uvw.

Weitere Verzeichnisse sind alphabetisch zu sortieren und besonders sorgfältig anzufertigen.
% Vorlage für WABs der Provadis Hochschule
% basiert auf "Universität Ulm Praktikumsbericht Vorlage"
% von Max Sch.
% CC BY 4.0
%
% 03_Methodik.tex

\chapter{Methodisches Vorgehen bei der Untersuchung}

Hier ein Beispiel zur Darstellung einer Abbildung (siehe Abbildung~\ref{fig:chap3:provadis}).

\begin{figure}[htb]
    \centering
    \includegraphics[width=0.5\textwidth]{Bilder/Deckblatt/provadis-hochschule.pdf}
    \caption{Beispiel-Provadis-Logo}
    \label{fig:chap3:provadis}
\end{figure}

\section{Quellcode}
Kurze Codeauszüge können wie folgt eingebunden werden:

\begin{lstlisting}
    Das ist der Code.
\end{lstlisting}

Längere Auszüge können auch direkt als Datei eingebunden werden (Siehe Listing~\ref{code:chap3:python}).

\lstinputlisting[language=Python,caption=Etwas Python-Code,label={code:chap3:python},style=myStyle]{Code/03_python1.py}

% Vorlage für WABs der Provadis Hochschule
% basiert auf "Universität Ulm Praktikumsbericht Vorlage"
% von Max Sch.
% CC BY 4.0
%
% 04_Untersuchung.tex

\chapter{Durchführung der Untersuchung}

\begin{table}[h]
    \centering
    \begin{tabular}{| l c r |}
        \hline
        1 & 2 & 3 \\
        4 & 5 & 6 \\
        7 & 8 & 9 \\
        \hline
    \end{tabular}
    \caption{Eine einfache Tabelle}
    \label{tab:chap4:simpel}
\end{table}

Wichtig: Tabellen (siehe Tabelle~\ref{tab:chap4:simpel}) und Abbildungen (siehe Abbildung~\ref{fig:chap3:provadis}) sind im Text zu referenzieren!

% Vorlage für WABs der Provadis Hochschule
% basiert auf "Universität Ulm Praktikumsbericht Vorlage"
% von Max Sch.
% CC BY 4.0
%
% 05_Ergebnisse.tex

\chapter{Ergebnisse der Untersuchung und Diskussion}
% Vorlage für WABs der Provadis Hochschule
% basiert auf "Universität Ulm Praktikumsbericht Vorlage"
% von Max Sch.
% CC BY 4.0
%
% 06_Fazit.tex


\chapter{Fazit und Ausblick}


\pagenumbering{Roman}

\printbibliography
%erstellt das Literaturverzeichnis

% Vorlage für Praktikumsberichte
%
% A1_KI.tex
%

\addchap{KI-Verzeichnis}

Jede Nutzung von KI ist zu dokumentieren. Dazu wird hinter dem Literaturverzeichnis ein separates KI-Verzeichnis eingefügt, das alle KI-generierten Inhalte, die eingesetzten Systeme, die verwendeten \emph{Prompts} sowie die weitere Verwendung des Outputs der KI transparent macht.
Bei der mehrfachen Verwendung eines Systems werden die Einträge durchnummeriert.
Die Reihenfolge der Einträge entspricht der Reihenfolge der Verwendung im Text.
Das KI-Verzeichnis ist \textbf{verpflichtend}.


\begin{table*}[h]
    \renewcommand{\arraystretch}{2}
    \centering
    \begin{tabular}{p{0.15\textwidth} p{0.5\textwidth} p{0.25\textwidth}}
        \toprule
        \large System&\large Prompt&\large Verwendung\\
        \midrule
        ChatGPT 1 & What criteria should I use to select a leader? & weiterentwickelt \\
        ChatGPT 2 & Schreibe einen Text, in dem die folgenden Themen behandelt werden: Personalmarketing und seine Bedeutung für ein Unternehmen – der Zusammenhang zum Employer Branding – die Auswirkungen der Personalmarketingstrategie auf das Recruiting.  & verändert: Passagen ausgelassen  \\
        ChatGPT 3 & Entwurf einer Gliederung für eine Hausarbeit zum Thema Recruiting  & unverändert \\
        Elicit 1& Which elements should be included in an Employer Branding Plan?  & Passagen überarbeitet  \\
        \bottomrule
    \end{tabular}
\end{table*}

Sofern keine KI verwendet wurde, enthält das Verzeichnis nur den Eintrag:
\begin{verse}
    Es wurde keine KI verwendet.
\end{verse}

% Vorlage für Praktikumsberichte
%
% Erklärung.tex
%
% Ehrenwörtliche Erklärung

\addchap{Ehrenwörtliche Erklärung}

Hiermit bestätige ich, dass ich die vorliegende Arbeit persönlich und selbständig verfasst und keine anderen als die angegebenen Quellen und Hilfsmittel verwendet habe. Alle Stellen, die wörtlich oder sinngemäß anderen Quellen entnommen wurden, sind als solche kenntlich gemacht. Die Zeichnungen, Abbildungen und Tabellen in dieser Arbeit sind von mir selbst erstellt oder wurden mit einem entsprechenden Quellennachweis versehen. Diese Arbeit wurde weder in gleicher noch in ähnlicher Form von mir an anderen Hochschulen zur Erlangung eines akademischen Abschlusses eingereicht.

\vspace{2cm}
Fankfurt, den \dotfill

\hspace{10cm} {\footnotesize \fullname}

\vspace{2cm}

Die ehrenwörtliche Erklärung ist mit der digitalen Unterschrift (Bild) des Studierenden zu versehen. Bei einem UPLOAD der Arbeit im HVS ersetzt der Authentifizierungsprozess die Unterschrift – diese ist dann optional. 

In jedem Fall muss unter dem Text der ehrenwörtlichen Erklärung Ort und Datum angegeben werden.


\end{document}
%beendet das Dokument

